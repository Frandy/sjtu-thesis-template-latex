%%==========================
%% chapter01.tex for SJTU Master Thesis
%% based on CASthesis
%% modified by wei.jianwen@gmail.com
%% version: 0.3a
%% Encoding: UTF-8
%% last update: Dec 5th, 2010
%%==================================================

%\bibliographystyle{sjtu2} %[此处用于每章都生产参考文献]
\chapter{这是什么}
\label{chap:what}

这是上海交通大学(非官方)硕士学位学位论文~\LaTeX~模板,当前版本是~\version。

\section{模板的来历}

最早的一版交大学位论文\LaTeX{}模板是一位热心的物理系同学制作的。
那份模板参考了自动化所学位论文模板,使用了CASthesis.cls文档类,中文字符处理则采用当时最为流行的CJK--\LaTeX{}方案。
我根据交大研究生院对学位论文的要求进行了调整,完成了一份基本可用的交大\LaTeX{}学位论文模板。

但是,搭建一个CJK--\LaTeX{}环境并不简单,在Linux下配置环境和调用中文字体的流程,对我而言犹如梦魇一般。
在William Wang的建议下,我开始着手把模板移植到\XeTeX{}上。
他完成了最初的移植,谢谢他的出色工作,使得后续的工作比我预想中的顺利。

这个学位论文模板从我用它来完成学位学位论文以后,就没有更新过。
过了快两年,随着知识水平的提高,我又想断断续续再做一些完善模板的工作,因此对原有的硕士论文模板做了修改,并在此基础上做了交大学士和博士学位论文\LaTeX{}模板。

现在,交大学位论文\LaTeX{}模板的代码在github上维护,地址是:

	\url{https://github.com/weijianwen/sjtu-thesis-template-latex/}

学士学位论文、硕士学位论文、博士学位论文分别在bachelor-thesis,master-thesis和phd-thesis分支中维护。
从下面的链接中可分别获得做新交大学士、硕士、博士模板zip压缩包。

\begin{itemize}
	\item 交大学士学位论文模板 \url{https://nodeload.github.com/weijianwen/sjtu-thesis-template-latex/zipball/bachelor-thesis}
	\item 交大硕士学位论文模板 \url{https://nodeload.github.com/weijianwen/sjtu-thesis-template-latex/zipball/master-thesis}
	\item 交大博士学位论文模板 \url{https://nodeload.github.com/weijianwen/sjtu-thesis-template-latex/zipball/phd-thesis}
\end{itemize}

欢迎大家使用交大学位论文模板!你可以通过如下的途径反馈模板使用过程中遇到的问题:\href{https://github.com/weijianwen/sjtu-thesis-template-latex/issues}{开issue}
、\href{https://bbs.sjtu.edu.cn/bbsdoc?board=TeX_LaTeX}{水源LaTeX版}发帖,或者是给\href{mailto:weijianwen@gmail.com}{我}发送邮件---你可能需要好几天才能收到我的邮件回复。

\section{模板说明}
\subsection{模板特性}
\label{sec:features}

这个模板使用的中文解决方案是~\XeTeX/\LaTeX~。
参考文献使用~BibTeX~处理,可以生成符合国标~GBT7714~风格的参考文献列表。
模板在Windows,Linux和Mac OS X下测试通过,更详细的系统要求请参考\ref{sec:requirements}。

模板的外观表现和功能都放在~sjtumaster-xetex.cls~和~sjtumaster-xetex.cfg~中,在对外观进行细微调整时,只需要更新这两个文件,不需要对.tex源文件做修改。

最后,给出一个列表,罗列一下这个模板的功能要点:

\begin{itemize}
\item 使用~\XeTeX~引擎处理中文;
\item 包含中文字符的源文件(.tex, .bib, .cfg),编码都使用UTF-8;
\item 使用~BibTeX~处理参考文献。参考文献表现形式(格式)受~.bst~控制,方便在不同风格间切换,目前生成的列表符合国标GBT7714要求;
\item 可以直接插入EPS/PDF/JPG/PNG格式的图像,并且\emph{不需要}~bounding box~文件(.bb)。
\item 模板的格式受~sjtumater-xetex.cls~和~sjtumaster-xetex.cfg~控制,方便模板更新和模板修改。
\end{itemize}

\subsection{系统要求}
\label{sec:requirements}

要使用这个模板协助你完成研究生学位论文的创作,下面的条件必须满足:

\begin{itemize}
\item 操作系统字体目录中有~TeX Gyre Termes~西文的:Regular, Italic, Bold, Bold Italic~四种~OTF~字体\footnote{TeX Gyre Termes~字体可以从\href{http://www.gust.org.pl/projects/e-foundry/tex-gyre/termes}{http://www.gust.org.pl/projects/e-foundry/tex-gyre/termes}下载。我也附带了一份~TeX.Gyre.Termes.Fonts.zip~在模板中,解压缩到字体目录后用~fc-cache -fv~刷新即可,用~fc-list~应该能看到。};
\item 操作系统字体目录中有~AdobeSongStd、AdobeKaitiStd、AdobeHeitiStd、AdobeFangsongStd~四款中文字体\footnote{Adobe~这四款中文~OTF~字体可以从~Adobe Reader~安装目录拿到。};
\item \TeX~系统有~\XeTeX~引擎;
\item \TeX~系统有~ctex~宏包;
\item 你有使用~\LaTeX~的经验。
\end{itemize}

你可以试着编译模板文件夹中自带的~test.tex~文件,看看你的~\TeX~系统是否满足上面的要求:

\begin{lstlisting}[basicstyle=\small\ttfamily, caption={编译测试文件test.tex}, numbers=none]
xelatex test.tex
\end{lstlisting}

如果编译出的~test.pdf~中能够:显示中英文内容、显示4幅图像、正确嵌入~AdobeSongStd~和~TeXGyreTermes~字体(通过PDF阅读器的“属性”查看)、并且看到了英文字符的连字(ligature)和\textsc{SmallCapital}特性,那么恭喜你,你的~\TeX~系统应该能够编译这个学位论文模板。

目前,我在手头的几个~\TeX~环境上都做过测试,MacTeX 2011, TeXLive 2011和C\TeX 2.9都能够顺利编译。在你到版上抱怨模板不能工作前,请确定你的~\TeX~系统能够编译前面的~test.tex~文件。欢迎大家把编译的情况直接\href{mailto:wei.jianwen@gmail.com}{反馈}给我(但我并不能保证能解决问题)。来信的内就写你是否顺利编译模板、错误提示、操作系统版本、\TeX~系统版本、xunicode.sty~版本和fontspec.sty版本。
 
\subsection{模板文件布局}
\label{sec:layout}

\begin{lstlisting}[basicstyle=\small\ttfamily,caption={模板文件布局},label=layout,float,numbers=none]
  |-- diss.tex
  |-- README.pdf
  |-- sjtumaster-xetex.cfg
  |-- sjtumaster-xetex.cls
  |-- body
  |   |-- abstract.tex
  |   |-- app1.tex
  |   |-- app2.tex
  |   |-- chapter01.tex
  |   |-- chapter02.tex
  |   |-- conclusion.tex
  |   |-- projects.tex
  |   |-- pub.tex
  |   |-- resume.tex
  |   |-- symbol.tex
  |   \-- thanks.tex
  |-- figures
  |   \-- chap2
  |-- from.gkb.to.utf8.txt
  |-- GBT7714-2005NLang.bst
  |-- Makefile
  |-- reference
  |   |-- chap1.bib
  |   \-- chap2.bib
  |-- run.bat
  |-- run.sh
  \-- test.tex
\end{lstlisting}

你拿到手的模板文件大致会包含代码\ref{layout}所列的文件,乍看起来还是挺令人头大的。
并且,这还是“干净”的时候,等到真正开始处理的时候,会冒出相当多的“中间文件”,这又会使情况变得更糟糕。
所以,有必要对这些文件做一些简要说明。
看完这部分以后,你应该发现,其实你要关心的文件类型并没有那么多。

\subsubsection{格式控制文件}
\label{sec:format}

格式控制文件控制着论文的表现形式,包括以下几个文件:
sjtumaster-xetex.cfg, sjtumaster-xetex.cls~和~GBT7714-2005NLang.bst。
其中,``.cfg''和``.cls''控制论文主体格式,``.bst''控制参考文献条目的格式,

一般用户最好``忽略''格式控制文件的存在,不要去碰它们。
有其他格式需要,欢迎到板上发贴。
对于因为擅自更改格式控制文件出现的问题,我不一定能够解决。

\subsubsection{主控文件~diss.tex}
\label{sec:disstex}

主控文件~diss.tex~的作用就是将你分散在多个文件中的内容``整合''成一篇完整的论文。
使用这个模板撰写学位论文时,你的学位论文内容和素材会被``拆散''到各个文件中:
譬如各章正文、各个附录、各章参考文献等等。
在~diss.tex~中通过``include''命令将论文的各个部分包含进来,从而形成一篇结构完成的论文。
封面页中的论文标题、作者等中英文信息,也是在~diss.tex~中填写。
部分可能会频繁修改的设置,譬如行间距、图片文件目录等,我也放在了diss.tex中。
你也可以在diss.tex中按照自己的需要引入一些的宏包
\footnote{我对宏包的态度是:只有当你需要在文档中使用那个宏包时,才需要在导言区中用~usepackage~引入该宏包。如若不然,通过usepackage引入一大堆不被用到的宏包,必然是一场灾难。由于一开始没有一致的设计目标,\LaTeX~的各宏包几乎都是独立发展起来的,因重定义命令导致的宏包冲突屡见不鲜。}
。

大致而言,在~diss.tex~中,大家只要留意把``章''一级的内容,以及各章参考文献内容包含进来就可以了。
需要注意,处理文档时所有的操作命令{}\cndash{}xelatex, bibtex等,都是作用在~diss.tex~上,而\emph{不是}后面这些``分散''的文件,请参考\ref{sec:process}小节。

\subsubsection{论文主体文件夹body}
\label{sec:thesisbody}

这一部分是论文的主体,是以``章''为单位划分的。

正文前部分(frontmatter):中英文摘要(abstract.tex)。其他部分,诸如中英文封面、授权信息等,都是根据~diss.tex~所填的信息``画''好了,
不单独弄成文件。

正文部分(mainmatter):自然就是各章内容~chapter\emph{xxx}.tex~了。

正文后的部分(backmatter):附录(app\emph{xx}.tex);致谢(thuanks.tex);攻读学位论文期间发表的学术论文目录(pub.tex);个人简历(resume.tex)。
参考文献列表是``生成''的,也不作为一个单独的文件。另外,学校的硕士研究生学位论文模板中,也没有要求加入个人建立,所以我没有在~diss.tex~中引入resume.tex。

\subsubsection{图片文件夹~figures}
\label{sec:figuresdir}

figures~文件夹放置了需要插入文档中的图片文件(PNG/JPG/PDF/EPS),建议按章再划分子目录。

\subsubsection{参考文献数据库文件夹~reference}
\label{sec:bibdir}

reference~文件夹放置的是各章``可能''会被引用的参考文献文件。
参考文献的元数据,例如作者、文献名称、年限、出版地等,会以一定的格式记录在纯文本文件.bib中。
最终的参考文献列表是BibTeX处理.bib后得到的,名为~diss.bbl。
将参考文献按章划分的一个好处是,可以在各章后生成独立的参考文献,不过,现在看来没有这个必要。
关于参考文献的管理,可以进一步参考第\ref{chap:example}章中的例子。

\subsection{如何使用模板}
\label{sec:process}

模板的\LaTeX{}源文件需要用\XeTeX{}编译产生PDF文件。
我在此给出三种命令行下的编译方式:逐行手工执行、使用脚本、使用latexmk,大家可以根据自己的喜好,选择\emph{其中一种}完成工作。

\subsubsection{逐行手工执行}

模板使用~\XeTeX~引擎提供的~xelatex~的命令处理,作用于“主控文档”diss.tex。并且,可以省略扩展名。
在命令提示符下逐行敲入如下命令完成编译。

\begin{lstlisting}[basicstyle=\small\ttfamily, caption={手动执行编译过程}, numbers=none]
xelatex -no-pdf --interaction=nonstopmode diss
bibtex diss 
xelatex -no-pdf --interaction=nonstopmode diss 
xelatex --interaction=nonstopmode diss 
\end{lstlisting}

运行bibtex的时候会提示一些错误,猜测是~{{\sc Bib}\TeX}~对UTF-8支持不充分,一般不影响最终结果。留意因为拼写错误导致的``找不到文献错误''即可。

基本处理流程就是这样,一些~\LaTeX~排版的小例子可以参考第二章。

\subsubsection{使用脚本}

为方便使用,我把上面几条命令放到了两个脚本文件中。
Linux用户可以使用run.sh脚本,Windows用户可以使用run.bat。

\subsubsection{使用latexmk自动编译}

latexmk是一个类似于make的自动化工具,主要处理latex编译过程。它能够在编译过程中自动编译多次解决引用问题、解决与外部工具(如bibtex)的依赖关系、检查文件状态决定是否重新编译。模板自带了latexmk所需的RC文件``latexmkrc'',可通过如下的命令调用latexmk完成模板的编译。

\begin{lstlisting}[basicstyle=\small\ttfamily, caption={使用latemkx编译}, numbers=none]
latexmk -pv -pdf diss
\end{lstlisting}

\section{从CJK-\LaTeX{}转向\XeTeX}
\label{sec:whydvipdfm}

我习惯把v0.2a使用dvipdfmx编译的硕士学位论文模板称为``CJK-\LaTeX{}模板'',而这个使用\XeTeX{}引擎(xelatex程序)处理的模板则被称为``\XeTeX/\LaTeX{}模板''。
从CJK--LaTeX模板迁移到XeTeX\_LaTeX模板的好处有下:
\begin{enumerate}
\item[\large\smiley] 搭建~\XeTeX~环境比搭建CJK--\LaTeX~环境更容易;
\item[\large\smiley] 更简单的字体控制;
\item[\large\smiley] 完美支持PDF/EPS/PNG/JPG图片,不需要``.bb''文件;
\item[\large\smiley] 支持~OpenType~字体的复杂字型变化功能(通常只有字母字体才有,学术文章也暂时用不上);
\end{enumerate}

当然,这也是有代价的。由于~\XeTeX~比较新,宏包兼容性仍不是很完美(\LaTeX~也存在这样的问题),很多宏包仍然在“修正”,以便更好地协同工作。在我看来,使用这个XeTeX模板所必须付出的代价是:

\begin{enumerate}
\item[\large\frownie] 必须把你“古老的”~\TeX~系统更新为较新的版本。TeXLive 2010~和~CTeX 2.9~能够编译这份模板,而~TeXLive 2009~和~CTeX 2.8~对此则无能为力;
\item[\large\frownie] 需要花一些时间把你在老模板上的工作迁移到新模板上。
\end{enumerate}

第一条就看你如何取舍了,新系统通常意味着更好的兼容性,值得升级。而转换模板也不是什么特别困难的事情,可以这样完成:

\begin{enumerate}
\item 备份你要转换的源文件,以防你的工作成果丢失;
\item 将你原来的``.tex''和``.bib''文件"另存为"UTF-8编码的文件。iconv、vim、emacs、UEdit~等等工具都可以完成。WinEdt~对文件编码识别功能很差(到了v6.0还是如此),不推荐作为字符编码转换工具;
\item 将``.tex''源文件中的以$\backslash$开始的~\LaTeX~命令,用\verb|~|将其与中文字符分开,否则与之相连的中文字符也被认为是命令的一部分,造成错误;
\item 将~diss.tex~导言区中的内容替换为~XeTeX~模板~diss.tex~导言区的内容;
\item 将你对原先导言区的修改,小心翼翼地``合并''到新的导言区中;
\item 使用~XeTeX~模板中的~GBT7714-2005NLang.bst~替换原有的bst文件,新的bst文件只是将字符编码转换为UTF--8。
\item 删除~bouding box~文件``.bb'';
\item 使用本文\ref{sec:process}介绍的方法,重新编译文档;
\end{enumerate}

\section{硕士学位论文格式的一些说明}
\label{sec:thesisformat}

所有关于研究生学位论文模板的要求,我参考的都是下面这个教务处的网址
\href{http://www.gs.sjtu.edu.cn/policy/fileShow.ahtml?id=130}{《上海交通大学研究生学位论文格式的统一要求 》}。

可惜,这个网址没有给出具体可用的“模板文件”。
并且,``要求''中的一些要求也不仅合理,譬如,公式和公式编号之前要用……连接,实现起来困难,看起来也不美观,从来没有人这样用,所以无视之。
师兄师姐的学位论文也是我可以参考的“范本”,尽管这些范本也不是很规范。
我希望制作出的这个学位论文模板尽可能符合教务处的要求,如果有任何建议,欢迎提出!

这个模板是为``双面打印''准备的,也就是说,迎面页总是奇数页,新的一章将从奇数页开始,``迎面页''和``背面页''(或者说奇数页和偶数页)的左右页眉是相互颠倒的,奇数页和偶数页的左右页边距也会被颠倒。通过双面打印得到的学位论文就像一本正常的书。

你可以将~diss.tex~中设定文档类的语句改为:

\begin{quote}
  {\scriptsize\verb+\documentclass[cs4size, a4paer, cs4size, oneside, openany]{sjtumaster-xetex}+}
\end{quote}

这样,就变成了适合“单面打印”的论文,新的一章可以从偶数页开始。

关于页眉页脚。奇数页页眉为:左边``上海交通大学硕士学位论文'',右边:``章节名'';偶数页页眉为:左边``上海交通大学硕士学位论文'',右边:``论文题目''。每一章的内容按照排书的习惯,均从奇数页开始。

教务处要求参考文献必须符合~GBT7714~风格,学校明确提出使用这个标准而不是自己拍脑袋想出别的做法,应该算是谢天谢地了。使用这个模板,结合BibTeX,可以很方便地生成符合GB标准的参考文献列表。

\section{模板更新说明}
\label{sec:update}

我希望这个模板能够成为大家完成学位论文的助手。
我会在一段时间内(一个月?一年?),继续维护这个模板,修正其中的错误和不理想的地方。
我还计划向模板中添加常用的``例子'',譬如表格、公式、图片的排版,这也是我知识汇总的。

模板的版本号由这两部分构成:``数字''+``字母'',譬如第一个版本的版本号v0.1a。
当对模板的修改达到了我认为可以释放出一个新版本的时候,我会更新版本号,并将新版本放在版上。
版本号的变动分两种情况:

\begin{itemize}
\item 如果仅仅是增加模板中的例子,不对模板的格式控制文件做修改,那么版本号将在字母上增加,数字不变,例如,0.1a$\rightarrow$0.1b;
\item 如果对格式控制文件(sjtumaster-xetex.cls, sjtumaster-xetex.cfg, GBxxx.bst)做了修改,版本号将在数字上做变动,同时字母回归a,例如,0.1c$\rightarrow$0.2a。
\end{itemize}

至此,大家可以通过留意版本号的变化来判断模板更新的类型。
完整的更新记录可参考附录A.

不管怎么说,模板更新总是一件好事。
因为在对模板进行修改时,我总是最大限度的保证其兼容性。
如果``新的格式控制文件''产生的效果对你很有吸引力,那么不妨尝试一下。
应用新的格式控制文件是一件非常简单的事情:
你只要把原来的~sjtumaster-xetex.cls, sjtumaster-xetex.cfg, GBxxx.bst~覆盖(建议备份或者使用版本控制系统),重新编译一遍,应该就OK了。


