%\bibliographystyle{sjtu2} %[此处用于每章都生产参考文献]
\chapter{这是什么}
\label{chap:what}

这是上海交通大学硕士学位学位论文~\LaTeX~模板,当前版本是0.1a。
这不是上海交大官方出品的的~\LaTeX~硕士论文模板,是大家根据研究生学位论文格式要求制作出的~\LaTeX~模板。

\section{模板的来历}

笔者不才,占了水源~TeX\_LaTeX~版版二的位置,只因自己对~\TeX~了解有限,一直没能做出一个使用“文档类”(documentclass)的学位论文模板。
正当我为这个事情发愁的时候,事情出现了重大转机:
一位交大物理系的热心同学在~CASthesis~文档类的基础上制作了交大博士学位论文~\LaTeX~模板。
借此东风,将该模板稍作修整后,我把它发在板上,供大家使用。
目前,交大博士论文~\LaTeX~模板工作良好,希望能使更多同学收益。

最近,又有同学提出希望能制作出交大硕士学位论文模板。
我参考了教务处对研究生学位论文格式的要求,在``博士学位论文~\LaTeX~模板''的基础上,制作了这个``硕士学位论文~\LaTeX~模板''。

欢迎大家测试使用交大硕士学位论文~\LaTeX~模板,有任何问题或者建议欢迎到水源~TeX\_LaTeX~版发贴(推荐方式),
也可以通过邮件向我反映模板中与学校要求不相符的地方,或者其他可能是~Bugs~的问题,邮件主题中请加``latex''前缀\footnotemark[1]。
\footnotetext[1]{笔者的联系方式是:\href{mailto:}{wei.jianwen@gmail.com}。最初那位制作模板的同学——也就是那位热心的物理系同学,留下的联系方式是\href{mailto:yang_tao@sjtu.edu.cn}{yang\_tao@sjtu.edu.cn},不过似乎联系不上。笔者目前负责修正模板中与学校格式要求不符的地方,以及模板的Bug。一般的~\LaTeX~问题,仍推荐到版上讨论。}

再次感谢那位物理系的热心同学!

\section{模板说明}
\label{sec:fastguide}

\subsection{模板特性}
\label{sec:features}

这个模板基于~CASthesis--0.1j~文档类,中文解决方案是~CJK。
鉴于~dvipdfmx~支持的图像类型广(PDF/EPS/JPG/PNG)、工作稳定、输出的质量高,
所以这个模板使用 latex $\rightarrow$ dvipdfmx 进行处理。
参考文献建议使用~BibTeX~管理,可以生成符合国标~GBT7714~风格的参考文献列表。

CASthesis~文档类又是基于~ctexbook~文档类——ctexbook~处理中文时对~GBK~编码青睐有加(处理UTF-8编码时又得使用ctexbookutf8),
所以引用源文件编码应该使用~GBK。

这个模板在~Windows~和~Linux~下测试都通过了,所以”跨平台“应该不成问题。

模板的外观表现和功能都放在~CASthesis.cls~和~CASthesis.cfg~中,在对外观进行细微调整时,只需要更新这两个文件,不需要对.tex源文件做修改。
这也给模板更新带来了极大方便。

最后,给出一个列表,罗列一下这个模板的特性:

\begin{itemize}
\item 需要使用~latex $\rightarrow$ dvipdfmx~处理模板;
\item 建议使用~BibTeX~管理参考文献;
\item 包含中文的源文件请使用~GBK~编码,UTF-8编码将不能处理;
\item 可以直接插入EPS/PDF/JPG/PNG格式的图像——不需要都转换为~EPS~格式。但是除了EPS外,都需要先使用~ebb~生成边框描述文件.bb(Bounding Box);
\item 模板在 Windows Server 2003 (32bit) + CTeX 2.7.0.39下测试通过;在 Gentoo Linux + TeXLive 2008 下测试通过(需要ctex宏包)。
\item 模板的表现形式受~CASthesis.cls~和~CASthesis.cfg~控制,方便更新模板和修改表现形式。
\item 参考文献表现形式(格式)受.bst控制,方便在不同风格间切换,目前生成的列表符合国标GBT7714要求;
\end{itemize}

\subsection{系统要求}
\label{sec:requirements}

要使用这个模板协助你完成研究生学位论文的创作,你的~\TeX~系统应该能满足如下要求:

\begin{itemize}
\item 能够使用~CJK~方案处理中文文档;
\item \TeX~系统中有~ctex~宏包。
\item 你有使用~\LaTeX~的经验。
\end{itemize}

使用~CTeX~系统自然满足上面的要求。

 用~TeXLive~自己搭建~\TeX~系统的同学,可以从~CTeX~中拷ctex宏包过来用,我本人\emph{不推荐}使用~GoogleCode SVN~上的~ctex-kit~宏包
 \footnote{\url{code.google.com/p/ctex-kit/} SVN版的ctex-kit很强大,但和这个模板配合起来使用却不是那么理想。这个主页上还有
   \href{http://code.google.com/p/ctex-kit/wiki/SimpleChineseTemplates}{其他一些有用信息} }。
 ctex~宏包中还有一个~fd~文件夹,对~ctex~中使用的中文字体名称做了定义(font definition)。
 使用者应该留意其中的信息,把字体映射到自己搭建~TeX~时定义的字体名字上。
 
\subsection{模板文件布局}
\label{sec:layout}

\begin{lstlisting}[title={模板文件布局},label=layout,float,numbers=none]
  |-- CASthesis.cfg
  |-- CASthesis.cls
  |-- GBT7714-2005NLang.bst
  |-- diss.pdf
  |-- diss.tex
  |-- diss.bbl
  |-- cctspace.cfg
  |-- body
  |   |-- abstract.tex
  |   |-- app1.tex
  |   |-- app2.tex
  |   |-- chapter01.tex
  |   |-- chapter02.tex
  |   |-- chapter03.tex
  |   |-- pub.tex
  |   |-- resume.tex
  |   |-- symbol.tex
  |   \-- thanks.tex
  |-- figures
  |   \-- chap1
  |       |-- fig1.bb
  |       |-- fig1.jpg
  |       |-- fig2.bb
  |       |-- fig2.jpg
  |       |-- testeps.eps
  |       \-- ...
  \-- reference
      |-- chap1.bib
      \-- chap2.bib
\end{lstlisting}

你拿到手的模板文件大致会包含\ref{layout}所列的文件,乍看起来还是挺令人头大的。
并且,这还是“干净”的时候,等到真正开始处理的时候,会冒出相当多的“中间文件”,这又会使情况变得更糟糕。
所以,有必要对这些文件做一些简要说明。
看完这部分以后,你应该发现,其实你要关心的文件类型并没有那么多。

\subsubsection{格式控制文件}
\label{sec:format}

格式控制文件控制着论文的表现形式,包括以下几个文件:
CASthesis.cfg, CASthesis.cls~和~GBT7714-2005NLang.bst.
其中,.bst控制的是参考文献条目的表现形式,前面两个控制的是论文除参考文献以外的表现。
还有一个文件——cctspace.cfg,也算是格式控制文件,控制的是中英文的间隔(大约如此),
必须配合~C\TeX~套件下的~cctspace~程序使用。
我主要是在~Linux~下编写模板,cctspace部分是否工作没有尽心过确认,所以暂不使用cctsapce
\footnote{Word中有很好的机制来调整中英文字符的间距。\LaTeX~中目前只能手动在中英文字符之间加入~字符,即~CJKtilde。
  cctspace的作用之一是:处理.tex文档,在每个中英文字符间插入~字符。可惜CJKtilde的间隔缺乏弹性,看起来有些呆板。
  在~XeTeX~中,已经有很好的方法解决这个问题,可惜这个模板暂时无福享受这样的好处了}。

\emph{\large 总之,一般用户最好``忽略''格式控制文件的存在,不要去碰它们。
   模板有问题,有其他格式需要,欢迎到板上发贴。
   对于因为擅自更改格式控制文件出现的问题,我一概不负责 {\LARGE\Smiley}
 }

\subsubsection{主控文件~diss.tex}
\label{sec:disstex}

主控文件~diss.tex~的作用就是将你分散在多个文件中的内容``整合''成一篇完整的论文。
使用这个模板撰写学位论文时,你的学位论文内容和素材会被``拆散''到各个文件中:
譬如各章正文、各个附录、各章参考文献等等。
在~diss.tex~中通过``include''命令将论文的各个部分包含进来,从而形成一篇结构完成的论文。
封面页中的论文标题、作者等中英文信息,也是在~diss.tex~中填写。
一些不便于写入.cls的格式设置,
或是引入一些``个性化''的宏包,也可以在~diss.tex~中引入
\footnote{我对宏包的态度是:只有当你需要在文档中使用那个宏包时,才需要在导言区中用~usepackage~引入该宏包。
  如若不然,通过usepackage引入一大堆不被用到的宏包,必然是一场灾难。因为一开始没有一致的设计目标,\LaTeX~的各宏包几乎都是独立发展起来的。
  因为重定义命令导致的宏包冲突屡见不鲜。}。

大致而言,在~diss.tex~中,大家只要留意把``章''一级的内容,以及各章参考文献内容包含进来就可以了。
\emph{需要注意,处理文档时所有的操作命令latex, bibtex, dvipdfmx等,都是作用在~diss.tex~上,而\emph{不是}后面这些``分散''的文件,请参考\ref{sec:process}小节。}

\subsubsection{论文主体文件夹body}
\label{sec:thesisbody}

这一部分是论文的主体,是以``章''为单位划分的,还可以在分为正文前的部分(frontmatter),正文(mainmatter),正文后的部分(backmatter)。

正文前部分(frontmatter):中英文摘要(abstract.tex)。其他部分,诸如中英文封面、授权信息等,都是根据~diss.tex~所填的信息``画''好了,
不单独弄成文件。

正文部分(mainmatter):自然就是各章内容~chapter\emph{xxx}.tex~了,这部分无法自动生成{\LARGE\Smiley}

正文后的部分(backmatter):附录(app\emph{xx}.tex);致谢(thuanks.tex);攻读学位论文期间发表的学术论文目录(pub.tex);个人简历(resume.tex)。
参考文献列表是``生成''的,也不作为一个单独的文件。另外,学校的硕士研究生学位论文模板中,也没有要求加入个人建立,所以我没有在~diss.tex~中引入resume.tex。

\subsubsection{图片文件夹~figures}
\label{sec:figuresdir}

figures~文件夹放置了需要插入文档中的图片文件(PNG/JPG/PDF/EPS),建议按章再划分子目录。
除了图片文件外,还有一些.bb文件,这些文件用来说明图片的``边界''(Bouding Box),其作用以及生成方法可参考第\ref{chap:example}章中插入图片的例子。

\subsubsection{参考文献数据库文件夹~reference}
\label{sec:bibdir}

reference~文件夹放置的是各章``可能''会被引用的参考文献文件。
参考文献的元数据,例如作者、文献名称、年限、出版地等,会以一定的格式记录在纯文本文件.bib中。
最终的参考文献列表是BibTeX处理.bib后得到的,名为~diss.bbl。
将参考文献按章划分的一个好处是,可以在各章后生成独立的参考文献,不过,现在看来没有这个必要。
关于参考文献的管理,可以进一步参考第\ref{chap:example}章中的例子。


\subsection{处理过程}
\label{sec:process}

模板使用~latex $\rightarrow$ dvipdfmx~的方式进行处理,所有的命令都是作用于“主控文档”diss.tex,并且,可以省略扩展名。完整的处理流程是:
\begin{quote}
latex diss  \%第一次运行latex \\
bibtex diss  \%处理参考文献\footnote{运行bibtex的时候会提示一些错误,我也拿它没办法,还好,最终结果良好,\emph{留意因为拼错名字提示的``找不到文献错误''即可}。} \\
latex diss  \%再次运行latex \\
latex diss  \%正确插入参考文献 \\ 
dvipdfmx diss \%输出为PDF文档
\end{quote}

如果你是~C\TeX~用户,在WinEdt编辑器中,dvipdfmx对应的按钮应该是dvi2pdf。
WinEdt那一排按钮中,有个墨绿色的狮子头对应的是``Texify'',可以根据依赖关系帮你生成好.dvi。
得到.dvi文件以后,可以按``dvi2pdf''按钮生成pdf文件。
还有一个狮子头``TexifyPDF''调用的是pdflatex,无法配合这个模板工作,所以不能使用。

基本处理流程就是这样,一些~\LaTeX~排版的小例子可以参考第二章。
大家可以根据自己的喜好额外使用一些工具,譬如~autotools~或者是批处理。

\section{硕士学位论文格式的一些说明}
\label{sec:thesisformat}

所有关于研究生学位论文模板的要求,我参考的都是下面这个教务处的网址
\href{http://www.gs.sjtu.edu.cn/policy/fileShow.ahtml?id=130}{《上海交通大学研究生学位论文格式的统一要求 》}。

可惜,这个网址没有给出具体可用的“模板文件”。
并且,``要求''中的一些要求也不仅合理,譬如,公式和公式编号之前要用……连接,实现起来困难,看起来也不美观,从来没有人这样用,所以无视之。
师兄师姐的学位论文也是我可以参考的“范本”,尽管这些范本也不是很规范。
我希望制作出的这个学位论文模板尽可能符合教务处的要求,如果有任何建议,欢迎提出!

这个模板是为``双面打印''准备的,也就是说,迎面页总是奇数页,新的一章将从奇数页开始,``迎面页''和``背面页''(或者说奇数页和偶数页)的左右页眉是相互颠倒的,奇数页和偶数页的左右页边距也会被颠倒。通过双面打印得到的学位论文就像一本正常的书。

你可以将~diss.tex~中设定文档类的语句改为:

\begin{quote}
\verb+\documentclass{CASthesis}+
\end{quote}

这样,就变成了适合“单面打印”的论文,新的一章可以从偶数页开始。但是,“单面打印”似乎要所更多的工作——我在编译过程中遇到了一些错误,无伤大雅,但我必须按回车让latex强制完成剩下的工作。所以,我不推荐“单面打印”这种用法。

奇数页页眉为:左边``上海交通大学硕士学位论文'',右边:``章节名'';偶数页页眉为:左边``上海交通大学硕士学位论文'',右边:``论文题目''。每一章的内容按照排书的习惯,均从奇数页开始。

教务处要求参考文献必须符合GBT7714风格,学校明确提出使用这个标准而不是自己拍脑袋想出别的做法,应该算是谢天谢地了。
使用这个模板,结合BibTeX,可以很方便地生成符合GB标准的参考文献列表。

\section{模板更新说明}
\label{sec:update}

我希望这个模板能够成为大家完成学位论文的助手。
我会在一段时间内(一个月?一年?),继续维护这个模板,修正其中的错误和不理想的地方。
我还计划向模板中添加常用的``例子'',譬如表格、公式、图片的排版,这也是我知识汇总的。

模板的版本号由这两部分构成:``数字''+``字母'',譬如第一个版本的版本号v0.1a。
当对模板的修改达到了我认为可以释放出一个新版本的时候,我会更新版本号,并将新版本放在版上。
版本号的变动分两种情况:

\begin{itemize}
\item 如果仅仅是增加模板中的例子,不对模板的格式控制文件做修改,那么版本号将在字母上增加,数字不变,例如,0.1a$\rightarrow$0.1b;
\item 如果对格式控制文件(CASthesis.cls, CASthesis.cfg, GBxxx.bst)做了修改,版本号将在数字上做变动,同时字母回归a,例如,0.1c$\rightarrow$0.2a。
\end{itemize}
至此,大家可以通过留意版本号的变化来判断模板更新的类型。
完整的更新记录可参考附录A.

不管怎么说,模板更新总是一件好事。
因为在对模板进行修改时,我总是最大限度的保证其兼容性。
如果``新的格式控制文件''产生的效果对你很有吸引力,那么不妨尝试一下。
应用新的格式控制文件是一件非常简单的事情:
你只要把原来的~CASthesis.cls, CASthesis.cfg, GBxxx.bst~备份,再把新的版本复制到工作目录中,重新编译一遍,应该就OK了。

\section{意见反馈}
\label{sec:replay}

如果你对这个模板有任何问题或建议(前提是你认真读完这份文档后发现确实没有能帮助你的地方),欢迎到水源TeX\_LaTeX版上发贴反馈。
你的反馈将是我完善模板的动力!

如果你是~\TeX/\LaTeX~方面的专家,也可以直接给我发\href{mailto:wei.jianwen@gmail.com}{邮件},一起讨论改善模板的方法。
邮件的主题请加上latex字样。

\section{题外话:为什么使用dvipdfmx而不是pdflatex}
\label{sec:whydvipdfm}

我想这个模板使用~latex/dvipdfmx~进行处理,而不是使用“一次成形”的~pdflatex?
我想看了下面两个表以后,你心里应该会有个数。

\begin{table}[!hbp]
  \bicaption[tab:pdflatex]{pdflatex优缺点}{pdflatex优缺点}{Table}{features of pdflatex}
  \centering
  \begin{tabular}{r|l}
    \toprule
    优点: & 一步到位生成pdf;可直接插入PDF/PNG/JPG图像,不需要ebb计算边界; \\
    &  可直接插入PDF/PNG/JPG图像,不需要ebb计算边界。\\
    \midrule
    缺点: & 不支持直接插入EPS图像;\\
    & GBK编码的.tex文件需要配合ccmap宏包才能生成“可复制粘贴搜索的的中文PDF”;\\
    & PDF中文书签的问题一直不能很好解决;\\
    & 让这个模板在~pdflatex~方式下工作“正常”目前超出了笔者的能力。\\
    \bottomrule
  \end{tabular}
\end{table}


\begin{table}[!hbp]
  \bicaption[tab:latex]{latex/dvipdfmx优缺点}{latex/dvipdfmx优缺点}{Table}{features of latex/dvipdfmx}
  \centering
  \begin{tabular}{r|l}
    \toprule
    优点: & GBK/UTF-8编码的.tex文件都能生成“可复制粘贴搜索的的中文PDF”; \\
    & 可生成PDF中文书签 \\
    & 支持直接插入EPS/PDF/PNG/JPG图像; \\
    & 在模板中做了一些修改后,我能够让它在latex下正常工作。\\
    \midrule
    缺点: & 包含引用时,需要来回多次编译才能得到正确的结果;\\
    & 插入PDF/PNG/JPG时需要用ebb生成.bb文件描述“边界”; \\
    & ebb计算的PNG/JPG的边界不准确{\LARGE \Frowny} \\
    \bottomrule
  \end{tabular}
\end{table}


